% Options for packages loaded elsewhere
\PassOptionsToPackage{unicode}{hyperref}
\PassOptionsToPackage{hyphens}{url}
%
\documentclass[
]{article}
\usepackage{amsmath,amssymb}
\usepackage{iftex}
\ifPDFTeX
  \usepackage[T1]{fontenc}
  \usepackage[utf8]{inputenc}
  \usepackage{textcomp} % provide euro and other symbols
\else % if luatex or xetex
  \usepackage{unicode-math} % this also loads fontspec
  \defaultfontfeatures{Scale=MatchLowercase}
  \defaultfontfeatures[\rmfamily]{Ligatures=TeX,Scale=1}
\fi
\usepackage{lmodern}
\ifPDFTeX\else
  % xetex/luatex font selection
\fi
% Use upquote if available, for straight quotes in verbatim environments
\IfFileExists{upquote.sty}{\usepackage{upquote}}{}
\IfFileExists{microtype.sty}{% use microtype if available
  \usepackage[]{microtype}
  \UseMicrotypeSet[protrusion]{basicmath} % disable protrusion for tt fonts
}{}
\makeatletter
\@ifundefined{KOMAClassName}{% if non-KOMA class
  \IfFileExists{parskip.sty}{%
    \usepackage{parskip}
  }{% else
    \setlength{\parindent}{0pt}
    \setlength{\parskip}{6pt plus 2pt minus 1pt}}
}{% if KOMA class
  \KOMAoptions{parskip=half}}
\makeatother
\usepackage{xcolor}
\usepackage[margin=1in]{geometry}
\usepackage{color}
\usepackage{fancyvrb}
\newcommand{\VerbBar}{|}
\newcommand{\VERB}{\Verb[commandchars=\\\{\}]}
\DefineVerbatimEnvironment{Highlighting}{Verbatim}{commandchars=\\\{\}}
% Add ',fontsize=\small' for more characters per line
\usepackage{framed}
\definecolor{shadecolor}{RGB}{248,248,248}
\newenvironment{Shaded}{\begin{snugshade}}{\end{snugshade}}
\newcommand{\AlertTok}[1]{\textcolor[rgb]{0.94,0.16,0.16}{#1}}
\newcommand{\AnnotationTok}[1]{\textcolor[rgb]{0.56,0.35,0.01}{\textbf{\textit{#1}}}}
\newcommand{\AttributeTok}[1]{\textcolor[rgb]{0.13,0.29,0.53}{#1}}
\newcommand{\BaseNTok}[1]{\textcolor[rgb]{0.00,0.00,0.81}{#1}}
\newcommand{\BuiltInTok}[1]{#1}
\newcommand{\CharTok}[1]{\textcolor[rgb]{0.31,0.60,0.02}{#1}}
\newcommand{\CommentTok}[1]{\textcolor[rgb]{0.56,0.35,0.01}{\textit{#1}}}
\newcommand{\CommentVarTok}[1]{\textcolor[rgb]{0.56,0.35,0.01}{\textbf{\textit{#1}}}}
\newcommand{\ConstantTok}[1]{\textcolor[rgb]{0.56,0.35,0.01}{#1}}
\newcommand{\ControlFlowTok}[1]{\textcolor[rgb]{0.13,0.29,0.53}{\textbf{#1}}}
\newcommand{\DataTypeTok}[1]{\textcolor[rgb]{0.13,0.29,0.53}{#1}}
\newcommand{\DecValTok}[1]{\textcolor[rgb]{0.00,0.00,0.81}{#1}}
\newcommand{\DocumentationTok}[1]{\textcolor[rgb]{0.56,0.35,0.01}{\textbf{\textit{#1}}}}
\newcommand{\ErrorTok}[1]{\textcolor[rgb]{0.64,0.00,0.00}{\textbf{#1}}}
\newcommand{\ExtensionTok}[1]{#1}
\newcommand{\FloatTok}[1]{\textcolor[rgb]{0.00,0.00,0.81}{#1}}
\newcommand{\FunctionTok}[1]{\textcolor[rgb]{0.13,0.29,0.53}{\textbf{#1}}}
\newcommand{\ImportTok}[1]{#1}
\newcommand{\InformationTok}[1]{\textcolor[rgb]{0.56,0.35,0.01}{\textbf{\textit{#1}}}}
\newcommand{\KeywordTok}[1]{\textcolor[rgb]{0.13,0.29,0.53}{\textbf{#1}}}
\newcommand{\NormalTok}[1]{#1}
\newcommand{\OperatorTok}[1]{\textcolor[rgb]{0.81,0.36,0.00}{\textbf{#1}}}
\newcommand{\OtherTok}[1]{\textcolor[rgb]{0.56,0.35,0.01}{#1}}
\newcommand{\PreprocessorTok}[1]{\textcolor[rgb]{0.56,0.35,0.01}{\textit{#1}}}
\newcommand{\RegionMarkerTok}[1]{#1}
\newcommand{\SpecialCharTok}[1]{\textcolor[rgb]{0.81,0.36,0.00}{\textbf{#1}}}
\newcommand{\SpecialStringTok}[1]{\textcolor[rgb]{0.31,0.60,0.02}{#1}}
\newcommand{\StringTok}[1]{\textcolor[rgb]{0.31,0.60,0.02}{#1}}
\newcommand{\VariableTok}[1]{\textcolor[rgb]{0.00,0.00,0.00}{#1}}
\newcommand{\VerbatimStringTok}[1]{\textcolor[rgb]{0.31,0.60,0.02}{#1}}
\newcommand{\WarningTok}[1]{\textcolor[rgb]{0.56,0.35,0.01}{\textbf{\textit{#1}}}}
\usepackage{graphicx}
\makeatletter
\def\maxwidth{\ifdim\Gin@nat@width>\linewidth\linewidth\else\Gin@nat@width\fi}
\def\maxheight{\ifdim\Gin@nat@height>\textheight\textheight\else\Gin@nat@height\fi}
\makeatother
% Scale images if necessary, so that they will not overflow the page
% margins by default, and it is still possible to overwrite the defaults
% using explicit options in \includegraphics[width, height, ...]{}
\setkeys{Gin}{width=\maxwidth,height=\maxheight,keepaspectratio}
% Set default figure placement to htbp
\makeatletter
\def\fps@figure{htbp}
\makeatother
\setlength{\emergencystretch}{3em} % prevent overfull lines
\providecommand{\tightlist}{%
  \setlength{\itemsep}{0pt}\setlength{\parskip}{0pt}}
\setcounter{secnumdepth}{-\maxdimen} % remove section numbering
\ifLuaTeX
  \usepackage{selnolig}  % disable illegal ligatures
\fi
\IfFileExists{bookmark.sty}{\usepackage{bookmark}}{\usepackage{hyperref}}
\IfFileExists{xurl.sty}{\usepackage{xurl}}{} % add URL line breaks if available
\urlstyle{same}
\hypersetup{
  pdftitle={Homework 09},
  pdfauthor={yourname},
  hidelinks,
  pdfcreator={LaTeX via pandoc}}

\title{Homework 09}
\author{yourname}
\date{12 April, 2024}

\begin{document}
\maketitle

\hypertarget{overview}{%
\section{\texorpdfstring{\textbf{Overview}}{Overview}}\label{overview}}

This homework will involving fitting a linear model to data. Because you
should not fit models to data without examining variable relationships,
this assignment will involve elements taken from modules on
\href{https://gabrielcook.xyz/fods24/modules/18_linear_models.html}{visualizing
relationships},
\href{https://gabrielcook.xyz/fods24/modules/18_linear_models.html}{linear
models}, and potentially
\href{https://gabrielcook.xyz/fods24/modules/13_visualizing_data.html}{\{ggplot2\}}
if you desire.

\hypertarget{libraries-and-functions}{%
\section{\texorpdfstring{\textbf{Libraries and
Functions}}{Libraries and Functions}}\label{libraries-and-functions}}

\begin{Shaded}
\begin{Highlighting}[]
\NormalTok{R.utils}\SpecialCharTok{::}\FunctionTok{sourceDirectory}\NormalTok{(here}\SpecialCharTok{::}\FunctionTok{here}\NormalTok{(}\StringTok{"r"}\NormalTok{, }\StringTok{"functions"}\NormalTok{))}
\end{Highlighting}
\end{Shaded}

\begin{verbatim}
## -- Attaching core tidyverse packages ------------------------ tidyverse 2.0.0 --
## v dplyr     1.1.4     v readr     2.1.4
## v forcats   1.0.0     v stringr   1.5.1
## v ggplot2   3.5.0     v tibble    3.2.1
## v lubridate 1.9.3     v tidyr     1.3.0
## v purrr     1.0.2     
## -- Conflicts ------------------------------------------ tidyverse_conflicts() --
## x dplyr::filter() masks stats::filter()
## x dplyr::lag()    masks stats::lag()
## i Use the conflicted package (<http://conflicted.r-lib.org/>) to force all conflicts to become errors
## 
## Attaching package: 'vroom'
## 
## 
## The following objects are masked from 'package:readr':
## 
##     as.col_spec, col_character, col_date, col_datetime, col_double,
##     col_factor, col_guess, col_integer, col_logical, col_number,
##     col_skip, col_time, cols, cols_condense, cols_only, date_names,
##     date_names_lang, date_names_langs, default_locale, fwf_cols,
##     fwf_empty, fwf_positions, fwf_widths, locale, output_column,
##     problems, spec
\end{verbatim}

\begin{Shaded}
\begin{Highlighting}[]
\FunctionTok{library}\NormalTok{(tidyverse)}
\FunctionTok{library}\NormalTok{(GGally)}
\end{Highlighting}
\end{Shaded}

\begin{verbatim}
## Registered S3 method overwritten by 'GGally':
##   method from   
##   +.gg   ggplot2
\end{verbatim}

\begin{Shaded}
\begin{Highlighting}[]
\FunctionTok{library}\NormalTok{(easystats)}
\end{Highlighting}
\end{Shaded}

\begin{verbatim}
## # Attaching packages: easystats 0.7.1 (red = needs update)
## √ bayestestR  0.13.2    √ correlation 0.8.4  
## √ datawizard  0.10.0    x effectsize  0.8.6  
## √ insight     0.19.10   √ modelbased  0.8.7  
## √ performance 0.11.0    √ parameters  0.21.6 
## √ report      0.5.8     √ see         0.8.3  
## 
## Restart the R-Session and update packages with `easystats::easystats_update()`.
\end{verbatim}

\hypertarget{data}{%
\section{\texorpdfstring{\textbf{Data}}{Data}}\label{data}}

Read in data for your project so that you have numeric variables that
you could use to build a linear regression model to examine the fit of a
linear model.

\hypertarget{specifying-variables-1-pt}{%
\section{\texorpdfstring{\textbf{Specifying Variables} (1
pt)}{Specifying Variables (1 pt)}}\label{specifying-variables-1-pt}}

Specify the predictor(s) and outcome variables of the model and explain
why you are interested in examining these relationships for the project.
Don't get too complicated.

\hypertarget{visualize-variable-relationships-1-pt}{%
\section{\texorpdfstring{\textbf{Visualize Variable Relationships} (1
pt)}{Visualize Variable Relationships (1 pt)}}\label{visualize-variable-relationships-1-pt}}

Visualize the relationships between the predictor(s) and outcome
variable.

\hypertarget{transforming-predictors-2-pts}{%
\section{\texorpdfstring{\textbf{Transforming Predictors} (2
pts)}{Transforming Predictors (2 pts)}}\label{transforming-predictors-2-pts}}

Depending on the visual relationship the the predictor(s) and the
outcome variable, consider whether a transformation (linear or
otherwise) of your predictors would be appropriate.

If you have predictors that cannot have, or are unlikely to have values
of 0, you should modify your data frame to ensure a value of 0 is
possible so that you can interpret the y-intercept. There are variable
ways to do this. For example, you could simply \emph{center} a variable
by subtracting each value from the mean. You could \emph{standardize}
the values (e.g., {[}x - mean{]} / sd ) so that the mean is 0 and unit
change is in terms of a singe standard deviation. Some statisticians
(e.g., Gelman and Hill (2006)) argue to scale the data by two standard
deviations (e.g., {[}x - mean{]} / {[}2 * sd{]} ).

\hypertarget{fitting-the-linear-model-1-pt}{%
\section{\texorpdfstring{\textbf{Fitting the Linear Model} (1
pt)}{Fitting the Linear Model (1 pt)}}\label{fitting-the-linear-model-1-pt}}

Use \texttt{lm()} to fit the model and assign the model to an object so
you can inspect it to answer questions about it. For now, stick with an
additive (aka main-effects) model rather than a model with an
interaction. There is an interaction model option as a bonus.

\hypertarget{determining-the-overall-model-fit-2-pts}{%
\section{\texorpdfstring{\textbf{Determining the Overall Model Fit} (2
pts)}{Determining the Overall Model Fit (2 pts)}}\label{determining-the-overall-model-fit-2-pts}}

Determine how well the linear model fits your data. Because your project
will involve making some statements about the model fit, in a sentence
or more, explain the how well the model fits the data making sure to
include the information needed to make that assessment.

\textbf{Interpret/Explain:}

\hypertarget{interpreting-predictor-coefficients-2-pts}{%
\section{\texorpdfstring{\textbf{Interpreting Predictor Coefficients} (2
pts)}{Interpreting Predictor Coefficients (2 pts)}}\label{interpreting-predictor-coefficients-2-pts}}

Determine how well the predictors themselves account for the outcome
variable and provide an interpretation based on the necessary
information.

\textbf{Interpret/Explain:}

\hypertarget{examining-the-model-performance-1-pt}{%
\section{\texorpdfstring{\textbf{Examining the Model Performance} (1
pt)}{Examining the Model Performance (1 pt)}}\label{examining-the-model-performance-1-pt}}

All model have assumptions. Interpreting models should be done within
the context of those assumptions. At very least, get some practice
checking some model diagnostics and assumptions using functions from the
\textbf{\{performance\}} library (e.g., \texttt{check\_outliers()},
\texttt{check\_normality()}, \texttt{check\_collinearity()}).

Based on your findings, provide an explanation about whether your
assumptions were upheld or potentially violated.

\hypertarget{bonus-option-2-pts}{%
\section{\texorpdfstring{\textbf{Bonus Option} (2
pts)}{Bonus Option (2 pts)}}\label{bonus-option-2-pts}}

\textbf{Interaction Model}

If you wanted to try an interaction model use the \texttt{*} operator
when building your function:
\texttt{lm(outcome\ \textasciitilde{}\ x1\ *\ x2)}. You could compare
this model to the additive model to see which is a better fit to your
data.

\textbf{Interpret/Explain:}

Finally, knit an \texttt{html} file and upload it to:
\url{https://claremontmckenna.app.box.com/f/026ed41483fb4a739f0ddfc7fbf8fd01}

\end{document}
